
\chapter{Background}

\section{Discourse}

\todo{Explain what discourse is.}

\subsection{Shallow Discourse Parsing}

\todo{Explain what shallow discourse parsing is.}


\subsubsection{Discourse connectives}

See Table \ref{tbl:sense-hierarchy} for sense hierarchy.

\subsubsection{Explicit vs implicit discourse connectives} \label{sec:implexpl}

\begin{figure}
    \centering
    \begin{outline}
        \1 TEMPORAL
            \2 Asynchronous
            \2 Synchronous
        \1 CONTINGENCY
            \2 Cause
            \2 Pragmatic Cause
            \2 Condition
            \2 Pragmatic Condition
        \1 COMPARISON
            \2 Contrast
            \2 Pragmatic Contrast
            \2 Concession
            \2 Pragmatic Concession
        \1 EXPANSION
            \2 Conjunction
            \2 Instantiation
            \2 Restatement
            \2 Alternative
            \2 Exception
            \2 List
    \end{outline}
    \caption{Two first levels of sense hierarchy.}
\end{figure}


It is not always possible to consider implicit discourse relations simply as explicit relations with the connective removed.

\begin{exe}
\ex I want to go to New York, but I already booked a flight. (I am not going to New York.)\label{example:notny}
\ex I want to go to New York, so I already booked a flight. (I am going to New York.)\label{example:ny}
\end{exe}

Here, we have either a COMPARISON.Contrast, or a CONTINGENCY.Cause relation depending on what form the sense takes. If we were to remove the connective token, our linguistic intuition tells us to default for the meaning of Example \ref{example:ny}, that is, CONTINGENCY.Cause. How can we use this to our advantage when classifying implicit connectives? (Honest question, still don't know.)

\section{Related work}

Up until this task little focus had been given to this topic from a parsing perspective, something which reflected the work in the final contributions: all papers built upon the same discourse parser as presented by \citep{lin_pdtbstyled_2014} with little variance in the form of alternative architectures. Given the complexity of the task this is not surprising, since this allows contributors to work within a focus area. \citep{lin_pdtbstyled_2014} is the first PDTB-styled end-to-end discourse parser with a parsing pipeline that closely reflects the annotation pipeline of PDTB. We will have a closer look at the components of the parser in Section \ref{sec:lin_parser}.

\todo{Add a brief overview of results from CoNLL 2015.}
