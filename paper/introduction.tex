\chapter{Introduction}

Despite the progress in Natural Language Processing (NLP), one assumption remains steadfast: sentences are still treated as context-less atomical units. Machine Translation systems rarely look outside of its local context for translation clues, sentiment analysis modules have trouble dealing with juxtapositions, and information extraction systems often limit extraction to the immediate sentence. While this may seem like an odd assumption to make, it has been difficult to release this constraint from the equations; the increasing recall has rarely been worth the much faster decreasing precision.

Lately, it has come to the attention that we might have started to reach a convergence for certain tasks if we do not start to look into the inherent structure between sentences within documents. That is, how does one text unit relate to the another? These \emph{discourse relations}, as they often are called, is the main target for \emph{discourse parsing}.

Take for instance:

\begin{exe}
\ex \emph{Boeing would make cost-of-living adjustments projected to be 5 for each year of the contract} \underline{though} \textbf{the union has called the offer insulting}.\label{exemple:boeing}
\end{exe}

The connective unit \emph{though} in Example \ref{exemple:boeing} can be analyzed as a binding block between the italized unit and the bolded, expressing a \emph{contrastive comparison}. This is an example of a \emph{sense} of a discourse relation which labels each relation according to a given sense taxonomy. That said, not all connective units are necessarily as explicit:

\begin{exe}
\ex \emph{No wonder}. \textbf{We were coming down straight into their canal.}\label{exemple:nowonder}
\end{exe}

Here, despite the lack of a connective unit the two sentences clearly share a connection. The bolded sentence explains the previous one as \emph{the reason} for the lack of wonder. We will expand upon the difference between implicit and explicit relations in section \ref{sec:implexpl}, as well as how we can possibly classify the sense of a relation.


\section{Purpose}

The purpose of this thesis is to explore methods of improving sense classification in English Shallow Discourse Parsing. The main research question to explore is \emph{how can we use continuous semantic representations to increase performance in such a task?} In particular, \emph{what benefits can we get from applying neural network architectures with continuous semantic representations?}
