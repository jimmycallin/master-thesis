\chapter{Data}

For training we will be using the Penn Discourse Treebank (PDTB). PDTB follows the lexically grounded predicate-argument approach as proposed in Webber (2004). It covers the subset containing Wall Street Journal articles from the Penn Treebank, making up approximately one million tokens. When a connective explicitly appears, it will be syntactically connected to the \emph{Arg2} argument of the discourse structure. \emph{Arg1} is the other one. Due to \emph{Arg2} being syntactically bounded to connective, it is easy to automatically classify \emph{Arg2}.

PDTB annotates each structure with types of discourse relations according to a three level hierarchy as seen in Table \ref{tbl:sense-hierarchy}, where the first level is made up of four classes: TEMPORAL, CONTINGENCY, COMPARISON, EXPANSION. Each class has a non-obligatory second level of in total 16 types to provide a more fine-grained classification. Due to the third level being considered too fine-grained, it is ignored in this work.

Furthermore, we have some additional resources at our disposal:

\begin{itemize}
    \item Brown clusters from the RCV1 corpus.
    \item MPQA subjectivity lexicon.
    \item Skip-gram Neural Word Embeddings trained on 100 billion words from the Google News dataset.
    \item VerbNet 3.2.
\end{itemize}

\begin{table}
\begin{tabular}{lrr}
\toprule
{} &  connective\_token &     ratio \\
senses                                       &                   &           \\
\midrule
(Expansion.Conjunction,)                     &              7355 &  0.226064 \\
(Contingency.Cause,)                         &              4969 &  0.152728 \\
(Comparison.Contrast,)                       &              4521 &  0.138958 \\
(EntRel,)                                    &              4133 &  0.127032 \\
(Expansion.Restatement,)                     &              2640 &  0.081143 \\
(Temporal.Asynchronous,)                     &              2044 &  0.062825 \\
(Expansion.Instantiation,)                   &              1392 &  0.042785 \\
(Comparison.Concession,)                     &              1268 &  0.038973 \\
(Contingency.Condition,)                     &              1114 &  0.034240 \\
(Temporal.Synchrony,)                        &               985 &  0.030275 \\
(Comparison,)                                &               484 &  0.014876 \\
(Expansion.Alternative,)                     &               435 &  0.013370 \\
(Temporal.Asynchronous, Contingency.Cause)   &               148 &  0.004549 \\
(Temporal.Synchrony, Contingency.Cause)      &               113 &  0.003473 \\
(Temporal.Synchrony, Expansion.Conjunction)  &               109 &  0.003350 \\
(Expansion.Conjunction, Contingency.Cause)   &               106 &  0.003258 \\
(Expansion,)                                 &                96 &  0.002951 \\
(Expansion.Conjunction, Temporal.Synchrony)  &                92 &  0.002828 \\
(Temporal.Synchrony, Comparison.Contrast)    &                51 &  0.001568 \\
(Expansion.Conjunction, Comparison.Contrast) &                47 &  0.001445 \\
\bottomrule
\end{tabular}

\label{tbl:sense_frequency}
\caption{Frequency of sense labels in training data. Right now connective\_token is simply the frequency. I should change this name. Ratio is the frequency ratio.}
\end{table}
